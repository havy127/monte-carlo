\chapter{Kết luận và hướng phát triển}\label{ch:4}
\section{Kết luận}\label{sec:4.1}
Đề tài tập trung tổng hợp các gói, khai báo và xây dựng các câu lệnh để tạo thành một tập tin lớp (class file)
hỗ trợ định dạng bố cục cho các khoá luận tốt nghiệp và báo cáo, đề tài cũng đồng thời đưa ra những
hướng dẫn cơ bản để người dùng có thể cài đặt và nắm bắt được cách tạo một tập tin đầu vào LaTeX đơn giản,
giúp người dùng làm quen với việc chuyển hoá ý tưởng thiết kế thành cấu trúc logic, để tập trung hơn
vào nội dung và trình tự sắp xếp các nội dung của văn bản hơn là tốn quá nhiều thời gian cho định
dạng bên ngoài, kế thừa đúng với mục đích của những người sáng tạo ra LaTeX.\par
Tuy chỉ mới thành công trong việc định dạng cơ bản các thành phần của một báo cáo khoá luận, cũng như
chưa có những câu lệnh chặt chẽ và môi trường mới để giúp người dùng thuận tiện trong việc
thiết kế các bố cục khác nhau, do thời gian tìm hiểu có giới hạn, nhưng sản phẩm cũng đủ để trở thành nền tảng cho nhiều hướng
phát triển sau này.\par
Ngoài tập tin lớp, bản thân báo cáo này cũng được xem là một thành phẩm của đề tài, do được tạo ra hoàn toàn
nhờ sử dụng lớp \texttt{vlththesis} này. Đồng thời, báo cáo cũng đã tổng hợp các tài liệu, nguồn thông tin
cần thiết, phân tích đầy đủ cấu trúc của tập tin lớp, nhằm giúp những ai quan tâm có thể dễ dàng nắm bắt,
chỉnh sửa, bổ sung, phát triển thêm cho lớp và sử dụng được thêm đầy đủ tính năng của các
g tích hợp.\par
\section{Hướng phát triển của đề tài}\label{sec:4.2}
Do tính chất mã nguồn mở của LaTeX, số lượng gói và lớp hỗ trợ LaTeX ngày càng được phát triển thêm 
nhiều tính năng, bố cục, môi trường mới tối ưu hơn những gói cũ, việc tìm hiểu, cập nhật các gói mới và tối ưu hơn để tích
hợp vào tập tin lớp cũng là điều cần thiết. Hơn nữa, do chỉ mới được xây dựng, lớp cần phải trải qua một thời gian dài sử dụng
mới có thể biết được những thiếu xót cần bổ sung, vì vậy việc tham khảo thêm các lớp mới, tìm kiếm gói phù hợp
bù đắp thiếu xót cũng là một hướng phát triển\footnote{Tuy nhiên vẫn cần phải chú ý tới độ tương thích của chúng để tránh
xung đột khi tích hợp quá nhiều}.\par
Như đã nói ở trên, lớp vẫn còn thiếu những câu lệnh giúp người dùng linh hoạt hơn trong việc thiết kế bố cục, hiện nay,
lớp chỉ dừng ở mức cho ra văn bản theo đúng khuôn khổ định trược mà chưa thực sự cho phép người soạn thảo tham gia
vào quá trình thiết kế, tuy có thể thay đổi những thiết lập gói mặc định của lớp, nhưng nếu muốn thay đổi thứ tự 
các thành phần, người dùng cần phải thay đổi trong tập tin lớp, hoặc định nghĩa lại hoàn toàn câu lệnh, do đó cần phải
tìm hiểu thêm nhiều thủ thuật lập trình LaTeX, để phát triển thêm các câu lệnh linh hoạt hơn và các môi trường trình bày
thông tin mới. Trước mắt có thể phát triển cho lớp hỗ trợ tốt hơn cho việc tạo các văn bản in hai mặt giấy.\par
Đề tài còn có thể phát triển thêm để tìm hiểu sâu hơn về TeX, các macro và câu lệnh của nó. Thao tác với các câu lệnh và macro
ở mức TeX, sẽ giúp cho câu lệnh chặt chẽ và hạn chế lỗi nhiều hơn, tuy nhiên, do đây là nền tảng cấp thấp hơn LaTeX, nên các
cú pháp của câu lệnh khá khó đọc và chuyên sâu, đòi hỏi nhiều thời gian tìm hiểu, nghiên cứu.\par
Trong số các gói cần tìm hiểu, đáng chú ý nhất là \texttt{Tikz}, cho phép người dùng tạo (vẽ) các đối
tượng hình thể trong LaTeX (graphic element) như: đường thẳng, hình tròn, hình chữ nhật, đường cong,\dots Đây là một
gói mạnh mẽ và phức tạp hỗ trợ đắc lực cho các công việc thiết kế, lập đồ thị và nhiều ứng dụng khác, việc tìm
hiểu câu lệnh của gói này và tích hợp vào \texttt{vlththesis} sẽ mở ra các khả năng cho phép người dùng có thêm nhiều lựa chọn trang
trí, thiết kế khung viền, hay lập đồ thị, sơ đồ khối,\dots \par
Vấn đề font chữ vẫn chưa được đề cập trong đề tài này, hướng phát triển tiếp theo có thể tập trung vào tìm hiểu thêm
về các khái niệm bộ mã kí tự và các gói về font chữ để hạn chế những cảnh báo giải mã tiếng Việt còn tồn động trong
lớp. Các khái niệm trong LaTeX, như hộp (box) và khoảng cách (length) cũng nên được tìm hiểu nhằm phát triển nhiều hiệu
ứng cho các kí tự, hiện nay có nền tảng biến thể XeLaTeX và XeTeX tập trung về vấn đề font chữ này.\par
Các gói và lớp hiện nay được phát triển sử dụng ba công cụ được tích hợp trong các gói phân phối đó là lớp \texttt{ltxdoc},
gói \texttt{doc} và công cụ \texttt{docstrip}, các công cụ này giúp người viết có thể xây dựng lớp, gói và tạo văn bản
hướng dẫn, thông tin trong cùng một tập tin duy nhất, phần văn bản vừa đóng vai trò chú thích cho câu lệnh, vừa trở thành câu chữ trong văn
bản hướng dẫn khi được xử lý trực tiếp bằng trình soạn thảo hỗ trợ LaTeX, khi được xử lý qua công cụ \texttt{docstrip}, tập tin tích hợp
đó sẽ trở thành một tập tin \texttt{.sty} hay \texttt{.cls} thông thường \cite{latex-comp}. Các công cụ trên cung cấp người viết phương tiện để xây dựng
các gói lớn, giúp cho họ dễ bảo trì, phát triển, soạn thảo văn bản hướng dẫn trong cùng một loại tập tin mà không cần tạo nhiều
tập tin riêng với các tên mở rộng khác nhau cho các mục đích trên. Do quy mô của lớp \texttt{vlththesis} và thời gian cho phép có hạn, nên đề tài chưa có cơ hội tìm
hiểu thêm và ứng dụng các công cụ đó để xây dựng tập tin lớp này, việc tìm hiểu về \texttt{doc} và \texttt{docstrip} là hướng phát
triển cần thiết cho những ai muốn xây dựng các gói và lớp quy mô, có thể phân phối được trên \acrshort{ctan}.\par
Và cuối cùng là tìm hiểu thêm về cách kết nối với cộng đồng người dùng LaTeX, ứng dụng GitHub và \acrshort{ctan} để phân phối
sản phẩm, nhận đánh giá và đóng góp từ người dùng, hướng đến việc xây dựng, bảo trì lớp và gói sản phẩm theo đúng tinh thần của cộng đồng mã nguồn mở, tìm hiểu kĩ
càng về giấy phép \acrshort{lppl} để có hướng phát triển đúng đắn cho thành phẩm.\par 
