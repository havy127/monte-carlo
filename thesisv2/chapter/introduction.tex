\startintroduction
\fancyhead[L]{}
Ngày nay, ngoài các trình soạn thảo văn bản phổ biến, LaTeX cũng là một sự lựa chọn dành cho người soạn thảo được
tạo ra với triết lý hoàn toàn khác biệt so với các trình hiện hành. Nhận thấy hạn chế của chất lượng in ấn ở những
năm 1970, và việc người dùng tốn quá nhiều thời gian để định dạng thay vì tập trung soạn thảo, Donald E.Knuth đã phát
triển hệ thống TeX, và từ đó, Leslie Lamport xây dựng thành LaTeX, với mục đích giúp người dùng sử dụng câu
lệnh để việc thiết kế văn bản được thực hiện một cách tự động bởi hệ thống. \par 
Tuy xuất hiện đã lâu nhưng do không có tính trực quan vốn có của các trình soạn thảo văn bản thông thường cũng như đòi hỏi người sử dụng có khái niệm cơ bản, 
về ngôn ngữ đánh dấu (markup language), cộng thêm việc nền tảng này chỉ lưu hành trong giới học thuật, nên LaTeX vẫn
chưa thực sự phổ biến đến những người dùng phổ thông (mặc dù đối tượng sử dụng ngày càng đa dạng).\par
Nhận thấy LaTeX thích hợp để tạo các văn bản có quy chuẩn rõ ràng, đồng thời nền tảng cho phép người dùng thiết kế bố cục và kiểu văn bản
cho riêng mình, đề tài này đã ra đời nhằm mục đích thiết kế, xây dựng một mẫu báo cáo khoá luận chuẩn trên nền LaTeX, định nghĩa các câu lệnh
mới để hỗ trợ những người dùng sau này có thể dễ dàng định dạng các báo cáo khoá luận mà không tốn nhiều thời gian vào việc thiết kế, canh chỉnh, thay vào đó
tập trung hơn vào nội dung và thành phần văn bản của mình, kế thừa đúng với tinh thần của những người sáng tạo ra LaTeX.\par
Tài liệu về LaTeX tuy đa dạng, nhưng lại có tính chuyên môn, đòi hỏi thời gian tìm hiểu và tổng hợp những tài liệu thật sự cần thiết, nhưng cũng nhờ
đó, tôi đã có thêm kĩ năng đọc hiểu, tìm kiếm thông tin, đồng thời hiểu thêm được các khái niệm, thao tác lập trình với
macro, cũng như tiếp cận và biết thêm được nhiều thủ thuật soạn thảo, trình bày văn bản theo ý mình sử dụng LaTeX, và đó
là những lý do tôi chọn đề tài này. Thông qua đề tài, ngoài việc xây dựng thành công một mẫu khoá luận, tôi cũng
muốn phổ biến sự tiện lợi trong việc soạn thảo các văn bản khoa học của LaTeX đến nhiều người hơn bằng việc giới thiệu, đưa
ra những hướng dẫn cơ bản và tổng hợp những nguồn tham khảo tin cậy cho hệ thống LaTeX này.\par
Báo cáo đề tài gồm bốn chương chính như sau:\par
\clearpage 
\begin{itemize}
 \item \textbf{Chương \ref{ch:1}: Tổng quan về LaTeX.} Giới thiệu khái niệm của LaTeX và lịch sử hình thành của hệ thống, đồng thời giới
 thiệu sơ lược về trình soạn thảo hỗ trợ LaTeX.
 \item \textbf{Chương \ref{ch:2}: Soạn thảo văn bản trong LaTeX.} Hướng dẫn cách tải và cài đặt nền tảng LaTeX trên hai
 hệ điều hành Windows và Linux, đồng thời đưa ra những hướng dẫn cơ bản về cách soạn thảo văn bản bằng LaTeX, các khái niệm,
 thuật ngữ và câu lệnh cần nắm để dễ dàng hiểu được các tài liệu hướng dẫn LaTeX.
 \item \textbf{Chương \ref{ch:3}: Thiết kế định dạng văn bản riêng trong LaTeX.} Sẽ tập trung vào cách thức thiết kế các
 định dạng văn bản riêng trong LaTeX, từ đó tiến tới thiết kế bài báo cáo, luận văn, sau đó
 phân tích quy trình tạo và cấu trúc của tập tin (file) sản phẩm đề tài.
 \item \textbf{Chương \ref{ch:4}: Kết luận và hướng phát triển.} Đưa ra kết luận về kết quả thu được của đề tài này
 và đánh giá hướng phát triển của thành phẩm.
\end{itemize}
