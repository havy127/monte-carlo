\documentclass{beamer}
\usepackage{color}
\usepackage[utf8]{inputenc}
\usepackage[vietnamese]{babel}
\usetheme{AnnArbor}
\usecolortheme{beaver}
\usepackage{scrextend}

\title[Giải tích số tích phân Monte Carlo nhiều lớp bằng gieo điểm quan trọng]{Đại học Khoa học Tự nhiên\\
Khoa Vật lý Vật lý Kỹ thuật\\
Bộ môn Vật lý Tin học }
\author[Vật lý Tin học]{ \color[rgb]{0.20,0.60,1.0} \Large GIẢI TÍCH SỐ TÍCH PHÂN MONTE CARLO NHIỀU LỚP BẰNG GIEO ĐIỂM QUAN TRỌNG}
% (optional, for multiple authors)
 \institute[] % (optional)
{
  \inst{}%
  \textbf{CBHD}: TS. Nguyễn Chí Linh
  \and
  \inst{}%
  \textbf{SVTH}: Huỳnh Thị Hạ Vy
}
\date[12/07/2019]% (optional)
{}


\begin{document}
 
\frame{\titlepage}
\begin{frame}{Nội dung báo cáo}
\fontsize{13pt}{40pt}
\begin{enumerate}
\item Giới thiệu 
\item Tích phân Monte Carlo
\item Thuật toán Vegas
\item Kết luận và hướng phát triển
\end{enumerate}
\end{frame}

%introduce
\begin{frame}{Giới thiệu}\vspace{4pt}
\color[rgb]{0.20,0.60,1.0} {Giới thiệu đề tài}\\
\vspace{0.5em}
\color[rgb]{0.0,0.00,0.0}
Tích phân là phép tính quan trọng trong lĩnh vực giải tích. Phương pháp tích phân Monte Carlo
là phương pháp sử dụng số ngẫu nhiên để tính toán kết quả gần đúng của tích phân. 
Để giảm phương sai cho tích phân, thuật toán Vegas xử lý các bài toán tích phân đa chiều.
Thuật toán Vegas sử dụng hai phương pháp lấy mẫu có trọng tâm và phương pháp phân tầng để giảm phương sai tăng độ chính xác cho phép tính.
\end{frame}
%Monte Carlo Integrand

\begin{frame}{Tích phân Monte Carlo}\vspace{4pt}
  \textbf{Tích phân một chiều}
  \begin{align}
  I=\int_{a}^{b}{f(x)dx}
\end{align}

  Ước tính giá trị của tích phân: 
  \begin{align}
  E(I) = (b-a)\langle{f}\rangle = \frac{b-a}{N}{\sum_{i=0}^{M-1}{f(x_i)}}
\end{align}
Trong đó:\\
  \vspace{1em}
  $x_i$: các giá trị lấy ngẫu nhiên phân bố đều trong khoảng [a,b]\\
  M: tổng số lần lấy mẫu $x_i$\\
\end{frame}

\begin{frame}{Tích phân Monte Carlo}\vspace{4pt}
\textbf{Tích phân một chiều}\\
\vspace{4em}
  Phương sai: 
\begin{align}
  \sigma^2(E(I)) = \frac{V}{M}\sum_{i=1}^M{(f(x_i)-\langle{f}\rangle)^2}
\end{align}
Với số mẫu M thì phương sai giảm theo $\frac{1}{N}$. Suy ra $\sigma \sim \frac{1}{\sqrt{N}}$\\
\vspace{0.5em}
Trong đó: 
$V=\int_{a}^{b}{dx}$: thể tích mở rộng của tích phân nhiều chiều\\
\end{frame}

\begin{frame}{Tích phân Monte Carlo}\vspace{4pt}
  \textbf{Tích phân Monte Carlo với hàm phân bố xác suất bất kỳ}\\
  \vspace{4em}
    Phương sai: 
  \begin{align}
    \langle{f}\rangle=\frac{1}{M}\sum_{i=0}^{M-1}{\frac{f(x_i)}{p(x_i)}}
  \end{align}
  \vspace{0.5em}
  Trong đó: 
  $p(x_i)$: hàm phân bố của biến ngẫu nhiên $x_i$\\
  \end{frame}
% Thuật toán Vegas


\end{document}