\documentclass{vlththesis}
\geometry{
left=0.79in, right=0.7in, top=0.8in, bottom=1in,
}
\newlength{\mylengthnew}
\settowidth{\mylengthnew}{Thesis Name:\ }

\begin{document}
\pagestyle{empty}
\begin{center}
\large
\theUniversity\\
\theFaculty\\
\bfseries
\theDepartment
\end{center}
\begin{center}
\Large
Thesis Synopsis
\end{center}
\begin{flushleft}
Student Name: Trinh Tich Thien \hspace{2in} Student ID: 1313172\\
Thesis Name: \textbf{DESIGN BACHELOR THESIS AND REPORT TEMPLATE IN\\ \hspace{\mylengthnew}LaTeX}\\
Student Department: \theDepartment\\
Supervisor: Dr. Nguyen Chi Linh\\
Workplace: \theUniversity
\end{flushleft}
\begin{center}
\textbf{\large Abstract}
\end{center}

Nowadays, aside from conventional text processors or editors, LaTeX is an alternative choice for
authors with different fields of expertise. It also introduces a completely new philosophy, which differs
from many popular text editors, to its user and easily becomes the most used typesetting system among
academic environment. In 1977, Donald E.Knuth first developed the TeX system, which later received
further development to become what is known today as LaTeX, from Leslie Lamport. Its raison d'être was
(is and will always be) providing utilities for authors to leave the visual design to the typesetting
system, and instead, focus on the contents of their works.\par
Realizing that LaTeX is appropriate for composing text with clear and unified standards, and the system
itself allows users to design layouts and styles for custom types of documents, this thesis was conceived
with the purpose is to design, build a template for bachelor theses. This template will help students spend
less time in designing and formating the visual aspect of their theses, reports and concentrate more on
the contents and logical structure of components of their works, staying true to the ideal of LaTeX's creators.\par
This thesis incorporated necessary packages, configurations and definitions for new commands (or control sequences as Knuth calls them)
into a single class file, which will form a standard layout for bachelor theses and reports, the thesis's report also
includes introductory manual to guide users in creating LaTeX environment as well as composing a simple LaTeX input file, which
is then built into a human-readable document.\par
Due to time constraints, the produced class file still haves undesired flaws, and in need of
test-run to evaluate performance and any necessary adjustments. However, the product is enough
to become a foundation for future development; moreover, the thesis's report has detailed analysis
on the structure of this class file and includes essential documentation, which, combining with
the class file, will serve as a guideline and reference for anyone who wants to build their own class or
package for LaTeX.\par
This thesis's directions for future development include (for more directions, please refer to chapter 4 of the report):\par
\begin{itemize}
 \item Refer to other class files for new ideas, search for new packages to offset the shortcomings of
 this class.
 \item Define more flexible commands and new formatting environment, modify the produced class to better support
 documents that require double-sided printing.
 \item Study the concept of text encoding and font packages to suppress Vietnamese characters encoding warnings, that
 are still present when building LaTeX input file on this class.
 \item Study about tools and utilities that support writing large and complex classes or packages.
\end{itemize}






\end{document}
