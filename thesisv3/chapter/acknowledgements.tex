Đầu tiên, con xin gửi lời biết ơn đến mẹ, người đã thay thế vai trò người cha đã mất, cáng đáng cả gia đình và nuôi
dưỡng con nên người, con cũng xin cảm ơn dì Chính, người mà còn vẫn luôn coi như người mẹ thứ hai, chăm sóc con từng
miếng ăn, giấc ngủ và luôn coi con như con ruột của mình, công ơn của hai mẹ dành cho con không từ ngữ nào mà diễn tả được.\par
Em xin cảm ơn các thầy cô khoa Vật Lý - Vật Lý Kĩ Thuật, đã tận tâm truyền đạt kiến thức cho em trong những năm đầu đại học.
Em xin chân thành cảm ơn thầy cô của Bộ môn Vật Lý Tin Học, đã xây dựng bộ môn với các trang thiết bị hiện đại và sự nhiệt
tình, thân thiện của các thầy cô, giúp em có thể thoái mái học tập, nghiên cứu mà không cảm thấy căng thẳng, áp lực. 
Những lời chỉ bảo của thầy cô đã cho em những kiến thức cần thiết và quý báu cho định hướng của mình.\par
Và em cũng xin gửi lời cảm ơn tới thầy TS. Nguyễn Chí Linh, đã giới thiệu và hướng em vào đề tài này khi em không xác định
được hướng đi cho mình, thầy cũng dành thời gian đọc, chỉnh sửa và góp ý cho bài báo cáo này được hoàn thiện hơn. Đồng thời
tôi cũng muốn cảm ơn những người bạn ở Vật Lý Lý Thuyết, đã dành thời gian với tôi trong những ngày mới bước vào chuyên ngành,
thông qua những buổi nói chuyện đó, tôi mới lần đầu biết đến nền tảng LaTeX.\par
Và cuối cùng, tôi xin cảm ơn những người bạn, những người đàn em đã cùng đồng hành với tôi trong suốt bốn năm trên giảng
đường Đại học, cảm ơn vì những khoảng khắc trò chuyện vui vẻ giúp giải toả áp lực đã trở thành một phần kỉ niệm của đời sinh viên.\par~\par
\hfill
\begin{minipage}[H]{0.5\textwidth}
 \centering
 \textsl{TP. Hồ Chí Minh, tháng 1 năm 2018.}\\
 \vspace{1cm}
 Trịnh Tích Thiện 
\end{minipage}
